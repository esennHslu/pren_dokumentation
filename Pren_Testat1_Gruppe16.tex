\documentclass{article}

\usepackage[ngerman]{babel}
\usepackage{datetime}
\usepackage[a4paper,top=2cm,bottom=2cm,left=3cm,right=3cm,marginparwidth=1.75cm]{geometry}
\usepackage{amsmath}
\usepackage{graphicx}
\usepackage{float} % For forcing the position of elements
\usepackage[colorlinks=true, allcolors=blue]{hyperref}
\usepackage{fancyhdr} % For custom headers and footers
\usepackage{titlesec}  % For automatic new page at sections
\usepackage{pdflscape}
\usepackage{longtable}
\usepackage{array}

% Set the Date Format
\newdateformat{myformat}{\THEDAY{ten }\monthname[\THEMONTH], \THEYEAR}

% Ensure a new page is started before every section
\newcommand{\sectionbreak}{\clearpage}
\newcommand{\thedate}{\today}

% Set up fancyhdr to customize the footer
\pagestyle{fancy}
\fancyhf{} % sets both header and footer to nothing
\renewcommand{\headrulewidth}{0pt}
\fancyfoot[C]{Pren Testat 1 Gruppe 16 - \thedate} % Add title and date to the center of the footer
\fancyfoot[R]{\thepage} % Page number on the right



\title{Pren Testat 1 Gruppe 16}
\author{Moritz Dillier \and Robin Venetz \and Benjamin Kuster \and Fabio Haueter \and Jonas Zimmermann \and Enya Senn}
\date{\thedate}

\begin{document}
\maketitle

\tableofcontents 


\section{Zeitplan}
Siehe Dokument PREN\_Projektplan\_T16.


\section{Teammitglieder}
\begin{figure}[H]
	\centering
	\includegraphics[width=0.8\linewidth]{Images/Pren_Gruppe.png}
	\caption{Teammitglieder}
	\label{fig:enter-label}
\end{figure}


\section{Aufgabenstellung}
In diesem Projekt wird ein autonomes Fahrzeug entwickelt, das sich in einem vorgegebenen Netzwerk von Wegpunkten (Nodes) und Kanten selbstständig orientieren und den optimalen Weg vom Start- zum Zielpunkt finden muss. Dabei müssen mehrere Herausforderungen bewältigt werden. Hindernisse auf der Strecke müssen erkannt, aktiv aufgehoben und an derselben Stelle wieder korrekt platziert werden. Zusätzlich dürfen gesperrte Wegpunkte (Nodes), die während des Fahrprozesses durch Markierungen gekennzeichnet werden, nicht befahren werden. Entfernte Kanten, die im Voraus nicht bekannt sind, werden vollständig aus dem Netz entfernt und sind für das Fahrzeug unsichtbar, was eine dynamische Anpassung der Routenplanung erfordert. Eine weitere Herausforderung ist, dass die Strecken zwischen den Wegpunkten eine länge von 0.5 m bis 2 m haben können und somit können sich die Winkel zwischen den Wegen verändern.

\hfill \break
Die Aufgabe erfordert eine präzise Pfadplanung, Echtzeit-Sensorverarbeitung sowie die Fähigkeit, auf unerwartete Änderungen im Netzwerk zu reagieren. Ziel ist es, dass das Fahrzeug den kürzesten und effizientesten Weg innerhalb eines komplexen, sich verändernden Streckennetzes unter Beachtung der vorgegebenen Regeln autonom bewältigt.

\begin{figure}[H]
	\centering
	\includegraphics[width=0.6\linewidth]{Images/Pren_Skizze_Anforderungen.png}
	\caption{Aufgabenstellung Skizze}
	\label{fig:enter-label}
\end{figure}

\begin{landscape} % Begin landscape mode
	\section{Anforderungsliste}
		
	\begin{longtable}{>{\raggedright\arraybackslash}l>{\raggedright\arraybackslash}l>{\raggedright\arraybackslash}p{4cm}>{\raggedright\arraybackslash}p{15cm}>{\raggedright\arraybackslash}c}
		\caption{Anforderungsliste} \\
		\textbf{Nr.} & \textbf{M/F/W} & \textbf{Thema/Frage}     & \textbf{Erläuterung, Werte, Daten}                                                                                                          & \textbf{Verantwortlich} \\
		\\
		\textbf{1}   &                & \textbf{Gerät}          &                                                                                                                                              &                         \\
		1.1 & F & Bauweise & Das Gerät muss eine Eigenkonstruktion sein. Einzelne Systemkomponenten, wie z.B. Servos,
		Motoren, Mikrocontrollerboard, Kamera u.a. dürfen zugekauft und eingesetzt werden. Es
		dürfen Software-Komponenten/-Services von Fremd-Herstellern verwendet werden. & E, I, M  \\ 
		1.2          & M              & Geräteabmessung         & l: max. 30 cm, b: max. 30 cm, h: max. 80 cm                                                                                                  & M                       \\
		1.3          & M              & Maximales Gewicht        & max. 2kg                                                                                                                                     & M                       \\
		1.4          & M              & Kosten                   & Die Gesamtkosten in der ersten Phase dürfen nicht mehr als 200.- und insgesamt nicht mehr als 500.- sein.                                   &                         \\
		1.5          & F              & Nachhaltigkeit           & Die Nachhaltigkeitsziele 9 sowie 12 müssen eingehalten werden                                                                               & E, I, M                 \\
		1.6          & F              & Wahlschalter             & Das Fahrzeug muss einen Wahlschalter besitzen, mit welchem man zwischen den Zielen A, B und C wählen kann.                                  & E, I                    \\
		1.7 & F & Not-Aus & Das Fahrzeug muss einen immer zugänglichen Not-Aus Schalter besitzen, welcher der mechanisch-dynamische Prozess
		jederzeit sofort unterbricht. & E, I, M \\
		1.8          & M              & Aufbauzeit               & Das Fahrzeug muss innerhalb von 1 Minute auf der Markierung platziert, aufgebaut und betriebsbereit sein.                                    &                         \\
		\\
		\textbf{2}   &                & \textbf{Fahrt/Ablauf}    &                                                                                                                                              &                         \\
		2.1          & F              & Startbefehl              & Startknopf / Taster / Schalter                                                                                                               & E, I                    \\
		2.2          & M              & Autonomes Fahren         & Beim Befahren der Linie muss immer ein Teil des Fahrzeugs auf der Linie verbleiben (ca. 20mm breite Linie)                                   & E, I                    \\
		2.3          & M              & Wegpunkte                & Es gibt aufgeklebte Wegpunkte, von denen mind. 2 abgefahren werden müssen.                                                                  & I                       \\
		2.4          & M              & Fahrzeit                 & Die Fahrzeit darf maximal 4 Minuten betragen.                                                                                                &                         \\
		2.5          & F              & Hindernisse              & Hindernisse müssen erkannt, aufgehoben und an die selbe Stelle gestellt werden (20mm Toleranz). (Schneepflug/fortschieben nicht zulässig). & E, I, M                 \\
		2.6          & F              & Sperrungen               & Sperrungen müssen erkannt werden und dürfen nicht bewegt werden.                                                                           & E, M, I                 \\
		2.7          & F              & Entfernte Routen         & Entfernte Routen müssen erkannt und können nicht genutzt werden.                                                                           & E, I                    \\
		2.8          & M              & Zielposition             & Das Fahrzeug muss in einem Radius von 15 cm um den gewählten Zielpunkt zum Stehen kommen.                                                   & E, I                    \\
		2.9          & M              & Erreichen des Endes      & Das Erreichen der Zielposition muss visuell oder akustisch signalisiert werden.                                                              & E, I                    \\
		\\
		\textbf{3} & & \textbf{Randbedingungen} \\
		3.1          & F              & Wegpunkte                & Vollkreis, weiss, D = 7 - 12 cm                                                                                                              & Doz.                    \\
		3.2          & F              & Verbindungslinien Länge & 0.5 - 2 m                                                                                                                                    & Doz.                    \\
		3.3          & F              & Verbindungslinien Breite & 20 mm +/- 5 mm                                                                                                                               & Doz.                    \\
		\\
		3.4          & F              & Abmessung Hindernis      & 135 x 38 x 60 mm, je +/- 15 mm                                                                                                               & Doz.                    \\
		3.5          & F              & Gewicht Hindernis        & ca. 200 g                                                                                                                                    & Doz.                    \\
		3.6          & F              & Untergrund               & Als Untergrund dient der Boden im Foyer vor der Mensa                                                                                        & Doz.                    \\
		\\
		\textbf{4} & & \textbf{Simulation} \\
		4.1          & M              & Art                      & Kein physischer Prototyp                                                                                                                     & I                       \\
				
	\end{longtable}
	
	\subsection{Legende}
	\begin{table}[H]
		\centering
		\begin{tabular}{cl}
			M & Mussanforderung   \\
			F & Festanforderung   \\
			W & Wunschanforderung \\
		\end{tabular}
		\caption{Legende Anforderung}
		\label{tab:my_label}
	\end{table}
	 
\end{landscape} % End landscape mode

\begin{landscape} % Begin landscape mode
	\section{Technische Recherche}
	\renewcommand*{\arraystretch}{1.8}
	\begin{longtable}{>{\raggedright\arraybackslash}m{3cm}>{\raggedright\arraybackslash}m{3cm}>{\raggedright\arraybackslash}m{5cm}>{\raggedright\arraybackslash}m{5cm}>{\raggedright\centering\arraybackslash}m{1cm}>{\raggedright\arraybackslash}m{7cm}}
		\caption{Technische Recherche} \\
				
				
		\textbf{Funktion}                & \textbf{Möglichkeiten}          & \textbf{Vorteile}                                                                                & \textbf{Nachteile}                                                                          & \textbf{Quelle}                                                                                                                             & \textbf{Beschreibung}                                                                                                                                                                                                                                                                                                                                                                                                                                                                     \\
		\hline
		Linie erkennen                   & Infrarotsensor                   & Einfache erkennung der Linie                                                                     & Durch Lichteinstrahlung beeinflusst                                                         & \href{https://www.futurelearn.com/info/courses/robotics-with-raspberry-pi/0/steps/75899}{Link}                                              & Der Sensor sendet IR Licht aus und misst, wie viel Licht reflektiert wird.                                                                                                                                                                                                                                                                                                                                                                                                                \\
		\cline{2-6}
		                                 & Kamera                           & Flexibler / kann auch für Objekterkennung genutzt werden                                        & Komplizierter als ein Infrarotsensor. Reflektierendes Licht kann beeinflussen.              & \href{https://www.instructables.com/Line-Following-Robot-Using-Smartphones-Camera/}{Link}                                                   & Erkennung der Linie durch Farberkennung auf Bildern.                                                                                                                                                                                                                                                                                                                                                                                                                                      \\
		\cline{2-6}
		                                 & Farbsensor                       & Präzise Farbunterscheidung, Billig                                                              & Empfindlich gegen Lichteinstrahlung, ähnliche Farben können schlecht unterschieden werden & \href{https://robotics.stackexchange.com/questions/2491/how-are-color-sensors-used-for-line-following}{Link}                                & Erkennung der Linie durch Erkennung der Farbe und des Farbkontrasts.                                                                                                                                                                                                                                                                                                                                                                                                                      \\
		\hline
		Auf der Linie\break bleiben      & 1 Sensor                         & Geringe Kosten                                                                                   & Ort auf der Linie nicht bekannt, muss Zick Zack fahren                                      & \href{https://www.instructables.com/SIMPLE-LINE-FOLLOWER-ROBOTsingle-Sensor/}{Link}                                                         & Ein Motor wird stärker als der andere angetrieben. Sobald die Line verlassen wird, wird der andere stärker betrieben.                                                                                                                                                                                                                                                                                                                                                                   \\
		\cline{2-6}
		                                 & 2+ Sensoren                      & Immer bekannt, wo/wie man sich auf der Linie befindet                                            & Höhere Kosten                                                                              & \href{https://robotics.stackexchange.com/questions/2491/how-are-color-sensors-used-for-line-following}{Link}                                & Der Linie wird mit einem Sensor gefolgt. Wenn ein Sensor die Linie nicht mehr erkennt, wird gegengesteuert.                                                                                                                                                                                                                                                                                                                                                                               \\
		\cline{2-6}
		                                 & Kamera                           & Nur eine Kamera nötig                                                                           & Komplexität                                                                                &                                                                                                                                             & Mit einer Kamera kann der Verlauf und die Position der Linie erkannt werden und Anpassungen an der Fahrtrichtung vorgenommen werden.                                                                                                                                                                                                                                                                                                                                                      \\
		\hline
														
		Identifikation                   & Bilderkennung                    & Kann beide Objekte erkennen und differenzieren                                                   & Hohe Rechenleistung benötigt. ML Modell muss trainiert werden                              & \href{https://www.delftstack.com/de/howto/python/color-detection-opencv/}{Link}                                                             & Mittels Bilderkennung (bsp.: OpenCV) werden die Hindernisse erkannt                                                                                                                                                                                                                                                                                                                                                                                                                       \\
		\cline{2-6}
		                                 & Farbsensor                       & Einfache differenzierung der Hindernisse anhand ihrer Farbe                                      & Ähnliche Farben können schwierig zu unterscheiden sein                                    & \href{https://www.electronicshub.org/raspberry-pi-color-sensor-tutorial/}{Link}                                                             & Erkennung der Farben. Evenutell mit zwei unterschiedlich hohen Farbsensoren                                                                                                                                                                                                                                                                                                                                                                                                               \\
		\hline
		Distanzmessung                   & Ultraschallsensor                & Billig, einfache Implementation/Integration                                                      & Limitierte Distanz und Genauigkeit                                                          & \href{https://www.geeksforgeeks.org/distance-measurement-using-ultrasonic-sensor-and-arduino/}{Link}                                        & Messung der Distanz zum Objekt mittels Ultraschallwellen                                                                                                                                                                                                                                                                                                                                                                                                                                  \\
		\cline{2-6}
		                                 & LIDAR                            & Genaue Karte der Umgebung kann erstellt werden                                                   & Preis                                                                                       & \href{https://de.wikipedia.org/wiki/Lidar}{Link}                                                                                            & Funktioniert ähnlich wie ein Ultraschallsensor. Misst Distanzen jedoch mittels Laserstrahlen                                                                                                                                                                                                                                                                                                                                                                                             \\
		\hline
		Simulation der Funktionalitäten & Unity                            & In C\# programmierbar, Funktionalität kann bereits getestet werden                              &                                                                                             & \href{https://unity.com/de}{Link}                                                                                                           & Simuluation des Fahrzeuges in der GameEngine Unity. Es können Kameras und Sensoren simuliert werden. Auch können Lichteinflüsse simuliert werden.                                                                                                                                                                                                                                                                                                                                      \\
		\cline{2-6}
		                                 & Website                          & Einfach erreichbar, schnelles Prototyping, Cross, Platform                                       & Limitierte Performance, nur in JS programmierbar                                            & \href{https://threejs.org/}{Link}                                                                                                           & Mittels WebGL und einer Library wie three.js können 3D umgebungen im Webbrowser simuliert werden.                                                                                                                                                                                                                                                                                                                                                                                        \\
		\cline{2-6}
		                                 & UnrealEngine                     & Programmierbar, Logik kann bereits getestet werden                                               & Wenig Kenntnisse in C++                                                                     & \href{https://www.unrealengine.com/de}{Link}                                                                                                & Simuluation des Fahrzeuges in der GameEngine UnrealEngine. Es können Kameras und Sensoren simuliert werden. Auch können Lichteinflüsse simuliert werden.                                                                                                                                                                                                                                                                                                                               \\
		\hline
		Algorithmus der Wegfindung       & dijkstra algorithm               & Garantiert den kürzesten Weg in einem statischen Netzwerk.                                      &                                                                                             & \href{https://www.geeksforgeeks.org/dijkstras-shortest-path-algorithm-greedy-algo-7/}{Link}                                                 & Er könnte verwendet werden, um die kürzeste Strecke vom Startpunkt zum Ziel zu bestimmen, basierend auf dem Wege-Netzwerk. Bei Hindernissen oder gesperrten Knoten könnte der Algorithmus dynamisch die Route anpassen, indem er die entsprechenden Knoten ausschließt.                                                                                                                                                                                                               \\
		\cline{2-6}
		                                 & Allstar Algorithmus              & Effizienter als Dijkstra durch die Verwendung einer Heuristik, die den Zielpunkt berücksichtigt & Die Wahl der Heuristik beeinflusst die Effizienz stark.                                     & \href{https://www.simplilearn.com/tutorials/artificial-intelligence-tutorial/a-star-algorithm }{Link}                                       & Erweiterung des Dijkstra-Algorithmus und berücksichtigt zusätzlich eine Heuristik, um effizienter zu sein. Dieser Algorithmus könnte verwendet werden, um den optimalen Pfad zu berechnen, während er Hindernisse und Sperren berücksichtigt.                                                                                                                                                                                                                                        \\
		\hline
		Programierung des Raspberry PI   & Python                           & Weit verbreitet, portable, viele Libraries, häufig für ML verwendet                            & Permformance limitationen                                                                   & \href{https://threejs.org/}{Link}                                                                                                           & Python ist die bevorzugte Sprache für ML auf dem Raspberry Pi, da es eine riesige Auswahl an ML-Bibliotheken bietet. Zudem hat es eine einfache Syntax und eine starke Community-Unterstützung. Es ist auch leicht, Python-Programme mit den GPIO-Pins zu verbinden, was den Einsatz von Sensoren und Aktoren in ML-Projekten erleichtert.                                                                                                                                              \\
		\cline{2-6}
		                                 & C\#                              & Bereits bekannt, kann in Unity getestet werden, Cross-Platform, Performance                      & Wenig Libraries für GPIO, noch nicht sehr lange unterstützt                               & \href{https://learn.microsoft.com/de-de/dotnet/iot/deployment}{Link}                                                                        & C\# kann über .NET oder Mono auf dem Raspberry Pi verwendet werden. Es gibt ML-Frameworks wie ML.NET, die in C\# für ML-Aufgaben auf dem Raspberry Pi genutzt werden können. C\# bietet eine objektorientierte Struktur und ist ideal für komplexere, skalierbare ML-Anwendungen. Die Performance ist besser als bei Python, aber der Zugriff auf GPIO und Hardware für ML-Anwendungen ist eingeschränkter. GPIO: https://learn.microsoft.com/en-us/dotnet/iot/tutorials/gpio-input \\
		\cline{2-6}
		                                 & Java                             & Protable                                                                                         & Weniger verrbreitet, weniger Libraries für H7                                              & \href{https://threejs.org/}{Link}                                                                                                           & C++ wird oft für ML auf Systemen mit beschränkten Ressourcen wie dem Raspberry Pi verwendet, insbesondere wenn Leistung entscheidend ist. Bibliotheken wie TensorFlow Lite und OpenCV sind in C++ verfügbar und können für ML-Aufgaben eingesetzt werden. C++ bietet die beste Leistung und direkte Hardwarezugriffe, was es für komplexe ML-Anwendungen und Echtzeit-Prozesse ideal macht.                                                                                         \\
		\cline{2-6}
		                                 & C++                              & Performant, Low-level access, viele Libraries                                                    & Komplex, eigenes Memory-Management, wenig Kenntnisse                                        & \href{https://threejs.org/}{Link}                                                                                                           & Java ist über OpenJDK auf dem Raspberry Pi verfügbar und kann für Machine Learning eingesetzt werden, etwa mit Bibliotheken wie Deeplearning4j. Es bietet plattformunabhängige Unterstützung und eignet sich gut für vernetzte Anwendungen, die auf ML-Modelle zugreifen müssen. Die Performance ist besser als bei Python, aber schwächer als bei C++. Der Hardwarezugriff über GPIO ist jedoch eingeschränkt.                                                                 \\
		\hline
				  
		Kinematischer Aufbau des Arms    & Freiheitsgrade                   & 3–4 Achsen: x, y, z Rotationsachse                                                             & Gewicht, Komplexität                                                                       &                                                                                                                                             & Basisrotation: Drehen des Arms um die eigene Achse, Schultergelenk für vertikales Heben, Ellenbogengelenk für zusätzliche Bewegungsfreiheit                                                                                                                                                                                                                                                                                                                                            \\
		\hline
		Greifer                          & Zangengreifer                    & Zwei Finger für stabile und gleichmäßige Objekte                                              & Weniger flexibel für unregelmäßige Formen                                                & \href{https://www.automation-next.com/kollegeroboter/grundlagen/id-4-wichtige-fakten-rund-um-heben-und-greifen-mit-robotern-114.html}{Link} & Einfacher Mechanismus für standardisierte Objekte                                                                                                                                                                                                                                                                                                                                                                                                                                        \\
		\cline{2-6}
		                                 & Saugnapf                         & Einfach und effizient, ideal für glatte Oberflächen                                            & Abhängigkeit von Oberflächenqualität                                                     & \href{https://www.automation-next.com/kollegeroboter/grundlagen/id-4-wichtige-fakten-rund-um-heben-und-greifen-mit-robotern-114.html}{Link} & Vakuumgreifer für glatte und saubere Oberflächen                                                                                                                                                                                                                                                                                                                                                                                                                                        \\
		\cline{2-6}
		                                 & Magnetgreifer                    & Ideal für metallische Objekte                                                                   & Nur für metallische Objekte geeignet                                                       & \href{https://automationspraxis.industrie.de/handling/greifer-fuer-roboter-grundlagen-funktion-und-hersteller/}{Link}                       & Greifen mittels elektromagnetischer Kräfte                                                                                                                                                                                                                                                                                                                                                                                                                                               \\
		\cline{2-6}
		                                 & 3-Finger-Greifer                 & Flexibilität für unregelmäßige Formen                                                        & Teurer, komplexer                                                                           & \href{https://automationspraxis.industrie.de/handling/greifer-fuer-roboter-grundlagen-funktion-und-hersteller/}{Link}                       & Stabilere und flexiblere Griffigkeit                                                                                                                                                                                                                                                                                                                                                                                                                                                      \\
		\hline
		Materialwahl                     & Aluminium                        & Leicht und stabil                                                                                & Teurer                                                                                      &                                                                                                                                             & Ideal für die Struktur des Greifarms                                                                                                                                                                                                                                                                                                                                                                                                                                                     \\
		\cline{2-6}
		                                 & Kunststoff (ABS/PLA)             & Günstig und leicht                                                                              & Weniger stabil                                                                              &                                                                                                                                             & Geeignet für Prototypen oder Teile des Arms                                                                                                                                                                                                                                                                                                                                                                                                                                              \\
		\hline
		Antrieb                          & Servomotoren                     & Gut steuerbar, präzise Kraftkontrolle                                                           &                                                                                             &                                                                                                                                             & Genaue Positionierung und Kraftsteuerung                                                                                                                                                                                                                                                                                                                                                                                                                                                  \\
		\cline{2-6}
		                                 & Schrittmotoren                   & Präzise Positionierung                                                                          & Höherer Stromverbrauch, weniger effizient bei hohen Geschwindigkeiten                      &                                                                                                                                             & Besonders für Bewegungen mit präzisem Winkel geeignet                                                                                                                                                                                                                                                                                                                                                                                                                                   \\
		\hline
		Steuerung                        & Arduino/Raspberry Pi             & Beliebt und weit verbreitet                                                                      &                                                                                             &                                                                                                                                             & Mikrocontroller-Plattformen zur Steuerung von Motoren und Sensoren                                                                                                                                                                                                                                                                                                                                                                                                                        \\
		\cline{2-6}
		                                 & Sensoren                         & Ultraschallsensoren oder Kameras                                                                 &                                                                                             &                                                                                                                                             & Zur Hinderniserkennung oder Bildverarbeitung                                                                                                                                                                                                                                                                                                                                                                                                                                              \\
		\hline
		Bewegungsfreiheit und Reichweite & Teleskoparm                      & Erhöhte Reichweite                                                                              & Gewicht                                                                                     &                                                                                                                                             & Erweiterbar für größere Reichweiten                                                                                                                                                                                                                                                                                                                                                                                                                                                    \\
		\hline
		Sicherheits-überlegungen        & Kraftsteuerung                   & Vermeidung von Schäden bei fragilen Objekten                                                    &                                                                                             &                                                                                                                                             & Kontrolle der Greifkraft, um fragile Objekte nicht zu beschädigen                                                                                                                                                                                                                                                                                                                                                                                                                        \\
		\cline{2-6}
		                                 & Feedback-Sensoren                & Sensoren wie Druck- oder Berührungssensoren                                                     &                                                                                             &                                                                                                                                             & Sicherstellen, dass der Greifer das Objekt richtig erfasst hat                                                                                                                                                                                                                                                                                                                                                                                                                            \\
		\hline
		Antriebssystem                   & Räder                           & Einfach, effizient für ebene Flächen                                                           & Weniger geeignet für unebenes Gelände                                                     &                                                                                                                                             & Effizientes System für präzise Steuerung auf ebenen Flächen                                                                                                                                                                                                                                                                                                                                                                                                                            \\
		\cline{2-6}
		                                 & Kettenantrieb                    & Größere Kontaktfläche mit dem Boden, geeignet für unebenes Gelände                          & Keine hohen Geschwindigkeiten                                                               &                                                                                                                                             & Effizient für unebenes Gelände                                                                                                                                                                                                                                                                                                                                                                                                                                                          \\
		\hline
		Antriebsart der Räder           & Differentialantrieb              & Hohe Wendigkeit, kann auf der Stelle drehen                                                      & Stabilisierung schwieriger auf unebenem Gelände                                            & \href{https://www.pkwteile.de/blog/antrieb-awd-rwd-fwd-4wd-was-ist-der-unterschied}{Link}                                                   & Präzise Steuerung und Wendigkeit für enge Räume                                                                                                                                                                                                                                                                                                                                                                                                                                        \\
		\cline{2-6}
		                                 & Allradantrieb                    & Bessere Stabilität und Kontrolle auf unebenem Gelände                                          & Komplexere Steuerung und Konstruktion                                                       & \href{https://www.pkw.de/magazin/antriebskonzepte/}{Link}                                                                                   & Alle Räder werden von Motoren angetrieben, ideal für schwere Lasten                                                                                                                                                                                                                                                                                                                                                                                                                     \\
		\cline{2-6}
		                                 & Omni-Räder                      & Hohe Bewegungsfreiheit in alle Richtungen                                                        & Weniger Traktion, komplizierte Steuerung                                                    &                                                                                                                                             & Bewegung in alle Richtungen ohne Fahrzeugdrehung möglich                                                                                                                                                                                                                                                                                                                                                                                                                                 \\
		\hline
		Motorwahl                        & DC-Motoren                       & Einfach zu steuern, preisgünstig                                                                & Keine genaue Positionskontrolle                                                             & \href{https://www.omc-stepperonline.com/de/support/was-dc-getriebemotor-ist}{Link}                                                          & Kostengünstig und effizient, aber weniger präzise                                                                                                                                                                                                                                                                                                                                                                                                                                       \\
		\cline{2-6}
		                                 & Schrittmotoren                   & Sehr präzise Positionierung                                                                     & Höherer Stromverbrauch, weniger effizient bei hohen Geschwindigkeiten                      & \href{https://kem.industrie.de/elektromotoren/was-sind-schrittmotoren-welche-typen-gibt-es-und-wie-funktionieren-sie/}{Link}                & Besonders für Bewegungen mit präzisem Winkel geeignet                                                                                                                                                                                                                                                                                                                                                                                                                                   \\
		\cline{2-6}
		                                 & Getriebemotoren                  & Hohe Kraft bei niedriger Geschwindigkeit                                                         & Größerer Platzbedarf, langsamer als andere Antriebsarten                                  &                                                                                                                                             & Ideal für Anwendungen, bei denen hohe Kraft bei niedriger Geschwindigkeit benötigt wird                                                                                                                                                                                                                                                                                                                                                                                                 \\
		\hline
		Fahrzeugstabilität              & Gewichtsverteilung               & Batterien als Gegengewicht zur Stabilisierung                                                    &                                                                                             & \href{https://www.lernhelfer.de/schuelerlexikon/physik-abitur/artikel/fahrphysik}{Link}                                                     & Verteilung des Gewichts, um Kippen zu verhindern, insbesondere bei schweren Objekten                                                                                                                                                                                                                                                                                                                                                                                                      \\
		\cline{2-6}
		                                 & Radstand                         & Breiter Radstand für Stabilität, aber auch Wendigkeit                                          &                                                                                             &                                                                                                                                             & Radstand so wählen, dass das Fahrzeug stabil bleibt, aber auch noch wendig genug ist                                                                                                                                                                                                                                                                                                                                                                                                     \\
		\hline
		Zusätzliche Sensoren            & Ultraschall/IR-Sensoren          & Hinderniserkennung und Kollisionsvermeidung                                                      &                                                                                             &                                                                                                                                             & Verwendung von Ultraschall- oder Infrarotsensoren                                                                                                                                                                                                                                                                                                                                                                                                                                         \\
		\cline{2-6}
		                                 & Gyroskop / Beschleunigungssensor & Überwachung der Fahrzeugstabilität, verhindert Kippen                                          &                                                                                             &                                                                                                                                             & Gyroskop und Beschleunigungssensoren zur Stabilitätsüberwachung                                                                                                                                                                                                                                                                                                                                                                                                                         \\
		\hline
								                                
	\end{longtable}
\end{landscape} % End landscape mode

\listoffigures % Abbildungsverzeichnis
\listoftables % Tabellenverzeichnis

\end{document}