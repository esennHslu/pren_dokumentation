\documentclass{article}

\usepackage[ngerman]{babel}
\usepackage{datetime}
\usepackage[a4paper,top=2cm,bottom=2cm,left=3cm,right=3cm,marginparwidth=1.75cm]{geometry}
\usepackage{amsmath}
\usepackage{graphicx}
\usepackage{float} % For forcing the position of elements
\usepackage[colorlinks=true, allcolors=blue]{hyperref}
\usepackage{fancyhdr} % For custom headers and footers
\usepackage{titlesec}  % For automatic new page at sections
\usepackage{pdflscape}
\usepackage{longtable}

% Set the Date Format
\newdateformat{myformat}{\THEDAY{ten }\monthname[\THEMONTH], \THEYEAR}

% Ensure a new page is started before every section
\newcommand{\sectionbreak}{\clearpage}
\newcommand{\thedate}{\today}

% Set up fancyhdr to customize the footer
\pagestyle{fancy}
\fancyhf{} % Clear default settings
\fancyfoot[C]{Pren Testat 1 Gruppe 16 - \thedate} % Add title and date to the center of the footer
\fancyfoot[R]{\thepage} % Page number on the right



\title{Pren Testat 1 Gruppe 16}
\author{Moritz Dillier \and Robin Venetz \and Benjamin Kuster \and Fabio Haueter \and Jonas Zimmermann \and Enya Senn}
\date{\thedate}

\begin{document}
\maketitle

\tableofcontents 

\begin{abstract}
Abstrakt
\end{abstract}

\section{Einleitung}
Das ist die Einleitung.
\section{Problemstellung}
Das ist die Problemstellung.



\begin{landscape} % Begin landscape mode
    \section{Anforderungsliste}
    \begin{longtable}{llp{4cm}p{15cm}c}
        \textbf{Nr.} & \textbf{M/F/K} & \textbf{Thema/Frage} & \textbf{Erläuterung, Werte, Daten} & \textbf{Verantwortlich} \\
        \\
        \textbf{1} &  & \textbf{Gerät} & &   \\
        1.1 & F & Bauweise & Das Gerät muss eine Eigenkonstruktion sein. Einzelne Systemkomponenten, wie z.B. Servos,
        Motoren, Mikrocontrollerboard, Kamera u.a. dürfen zugekauft und eingesetzt werden. Es
        dürfen Software-Komponenten/-Services von Fremd-Herstellern verwendet werden. & E, I, M  \\ 
        1.2 & M & Geräteabmessung & l: 30 cm, b: 30 cm, h: 80 cm & M \\
        1.3 & M & Maximales Gewicht & 2kg & M \\
        1.4 & M & Kosten & Die Gesamtkosten in der ersten Phase dürfen nicht mehr als 200.- und insgesamt nicht mehr als 500.- sein. & \\
        1.5 & F & Nachhaltigkeit & Die Nachhaltigkeitsziele 9 sowie 12 müssen eingehalten werden & E, I, M \\
        1.6 & F & Wahlschalter & Das Fahrzeug muss einen Wahlschalter besitzen, mit welchem man zwischen den Zielen A, B und C wählen kann. & E, I \\
        1.7 & F & Not-Aus & Das Fahrzeug muss einen immer zugänglichen Not-Aus Schalter besitzen, welcher der mechanisch-dynamische Prozess
        jederzeit sofort unterbricht. & E, I, M \\
        1.8 & M & Aufbauzeit & Das Fahrzeug muss innerhalb von 1 Minute auf der Markierung platziert, aufgebaut und betriebsbereit sein. & \\
		\\

        \textbf{2} &  & \textbf{Fahrt/Ablauf} & &   \\
        2.1 & F & Startbefehl & Startknopf / Taster / Schalter & E, I \\
        2.2 & M & Autonomes Fahren & Beim Befahren der Linie muss immer ein Teil des Fahrzeugs auf der Linie verbleiben (ca. 20mm breite Linie) & E, I \\
        2.3 & M & Wegpunkte & Es gibt aufgeklebte Wegpunkte, von denen mind. 2 abgefahren werden müssen. & I \\
        2.4 & M & Fahrzeit & Die Fahrzeit darf maximal 4 Minuten betragen. & \\
        2.5 & F & Hindernisse & Hindernisse müssen erkannt, aufgehoben und an die selbe Stelle gestellt werden (20mm Toleranz). (Schneepflug/fortschieben nicht zulässig). & E, I, M \\
        2.6 & F & Sperrungen & Sperrungen müssen erkannt werden und dürfen nicht bewegt werden. & E, M, I \\
        2.7 & F & Entfernte Routen & Entfernte Routen müssen erkannt und können nicht genutzt werden. & E, I \\
        2.8 & M & Zielposition & Das Fahrzeug muss in einem Radius von 15 cm um den gewählten Zielpunkt zum Stehen kommen. & E, I \\
        2.9 & M & Erreichen des Endes & Das Erreichen der Zielposition muss visuell oder akustisch signalisiert werden. & E, I \\

        \\
        \textbf{3} & & \textbf{Randbedingungen} \\
        3.1 & F & Wegpunkte & Vollkreis, weiss, D = 7 - 12 cm & Doz. \\
        3.2 & F & Verbindungslinien Länge & 0.5 - 2 m & Doz. \\
        3.3 & F & Verbindungslinien Breite & 20 mm +/- 5 mm & Doz. \\
        \\
        3.4 & F & Abmessung Hindernis & 135 x 38 x 60 mm, je +/- 15 mm & Doz. \\
        3.5 & F & Gewicht Hindernis & ca. 200 g & Doz. \\
        3.6 & F & Untergrund & Als Untergrund dient der Boden im Foyer vor der Mensa & Doz. \\

		\\
        \textbf{4} & & \textbf{Simulation} \\
        4.1 & M & Art & Kein physischer Prototyp & I \\
        \label{tab:my_label}
    \end{longtable}
\end{landscape} % End landscape mode





\end{document}
